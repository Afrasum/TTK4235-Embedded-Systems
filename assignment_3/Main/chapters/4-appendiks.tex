\appendix

\section{Appendiks - Mer avanserte funksjoner}\label{4-appendiks}


Oppgaven dere nå løste var gjort med `brute force'. Dette funker, men det finnes andre elegante løsninger. Disse løsningene er basert på mønstergjennskjenning og dedikerte kilde- eller byggemapper.

Sett at man har et prosjekt som heter \verb|pizza|. Prosjektet består av kildefilene \verb|main.c, pizza_bread.c, pizza_sauce.c|, og \verb|pizza_topping.c|. For å holde oversikt er det lurt å ha en dedikert kildemappe der man lagrer kildekoden til prosjektet (vanligvis blir denne mappen kalt \verb|source|). I tillegg er det også ønskelig å ha en mappe for alle kompilerte \textit{artefakter} (vanligvis blir denne mappen kalt \verb|build|). 

I toppnivåmappen har man makefilen og det ferdige programmet:

\dirtree{%
.1 pizza.
.1 Makefile.
.1 build.
.2 main.o.
.2 pizza\_bread.o.
.2 pizza\_sauce.o.
.2 pizza\_topping.o.
.1 source.
.2 main.c.
.2 pizza\_bread.c.
.2 pizza\_sauce.c.
.2 pizza\_topping.c.
}

Det eneste man trenger for å bygge \verb|pizza| er kildefilene (\verb|source|) og makefilen. Det er derfor ønskelig å kunne automatisk opprette byggemappen (\verb|build|) om den ikke finnes. Dette kan gjøres med en \textit{order-only prerequisite}. Når man beskriver hvilke filer \verb|make| trenger for å bygge et mål, kan man bruke vertikal pipe (\texttt{|}) for å fortelle \verb|make| at avhengigheten kun trenger å eksistere:

\begin{lstlisting}
target : dependency_1 dependency_2 | order_only_1 order_only_2
         [commands to build target]
\end{lstlisting}

Alle avhengigheter som kommer etter \texttt{|}-tegnet vil kun bygges dersom de enten ikke allerede finnes, eller om man eksplisitt ber \verb|make| om å bygge det bestemte målet. Dette er nyttig for å automatisk opprette byggemappen om den ikke finnes. 

For å lage en regel om at kompilerte filer skal legges i byggemappen må man overstyre de infererte reglene ved å bruke mønstergjenkjenning. Mønstergjenkjenning brukes for å lage en generisk regel for hvordan en \verb|.c|-fil skal kompileres. For mønstergjenkjenning, brukes tegnet \verb|%|. Et eksempel på grunnleggende  mønstergje-\newline nkjenning kan ses under:

\begin{lstlisting}
%.o : %.c
        gcc -c $< -o $@
\end{lstlisting}

hvor \verb|$@| og \verb|$<| er automatiske variabler. \verb|$@| referer til målet som blir generert ved å kjøre regelen, mens \verb|$<| referer til første avhengighet i regelen. Akkurat denne regelen definerer byggeprosessen slik at en hvilken som helst \verb|.o|-fil blir generert ved å kompilere den tilsvarende \verb|.c|-filen med \verb|gcc|. Mønstergjenkjenning og \textit{order-only} avhengigheter kan kombineres elegant for å automatisk kompilere \verb|.c|-filene inn i den dedikerte byggemappen:


\begin{lstlisting}
build/%.o : %.c | build
        gcc -c $< -o $@
\end{lstlisting}

For å gjøre prosessen mer lettvint, er det greit å definere alle avhengighetene (\verb|main.c|, \verb|pizza_bread.c|, \verb|pizza_sauce.c|, og \verb|pizza_topping.c|) i en dedikert regel som referer til variablene. Dette gjøres ved å kombinere mønstergjenkjenning og substitusjon:

\begin{lstlisting}
 SOURCES := main.c pizza_bread.c pizza_sauce.c pizza_topping.c
 
 SRC := $(SOURCES:%c=source/%c)
\end{lstlisting}

Denne deklarasjonen vil ta alle .c filene fra variabelen \verb|SOURCES| og legge til mappeprefikset \verb|source|. Dermed trenger man ikke å skrive alle avhengighetene (\verb|main.c|, \verb|pizza_bread.c|, \verb|pizza_sauce.c|, og \verb|pizza_topping.c|) for hver enkel fil, men man kan bare bruke variabelen \verb|SOURCES|. Den komplette makefilen for \verb|pizza| består av følgende kode:


\begin{lstlisting}
SOURCES := main.c pizza_bread.c pizza_sauce.c pizza_topping.c

BUILD_DIR := build

OBJ := $(SOURCES:%.c=$(BUILD_DIR)/%.o)

SRC_DIR := source
SRC := $(SOURCES:%.c=$(SRC_DIR)/%.c)

CC := gcc
CFLAGS := -O0 -g3 -Wall -Werror

.DEFAULT_GOAL := pizza

pizza : $(OBJ)
    $(CC) $(OBJ) -o $@
    
$(BUILD_DIR) :
    mkdir $(BUILD_DIR)

$(BUILD_DIR)/%.o : $(SRC_DIR)/%.c | $(BUILD_DIR)
    $(CC) -c $< -o $@

.PHONY : clean
clean:
    rm -rf $(.DEFAULT_GOAL) $(BUILD_DIR)
\end{lstlisting}

